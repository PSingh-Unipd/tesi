% !TEX encoding = UTF-8
% !TEX TS-program = pdflatex
% !TEX root = ../tesi.tex

%**************************************************************
% Sommario
%**************************************************************
\cleardoublepage
\phantomsection
\pdfbookmark{Sommario}{Sommario}
\begingroup
\let\clearpage\relax
\let\cleardoublepage\relax
\let\cleardoublepage\relax

\chapter*{Sommario}
Il presente documento descrive il lavoro svolto durante il periodo di stage(dal 03/06/2018 al 03/08/2018), della durata di circa 320 ore, dal laureando Singh Parwinder presso l'azienda Nextep alla sede di Cittadella.
\\
In questo documento verrano descritte in dettaglio l'analisi dei requisiti, la progettazone, l'implementazione e la validazione dell'applicazione CS-template. L'applicazione in questione è stato realizzato utilizzanod le tecnolgie web inovvative sia per quanto riguarda lato frant-end dell'applicazione che quello back-end.\\
L'intero lavoro è stato svolto in ambiente Linux Ubuntu 18.04 LTS. Tutti i diagrammi delle classi, dei package e dei casi d’uso (presenti nei Capitoli 3 e
4) sono conformi allo standard UML 2.0. Per realizzarli è stato usato il software Astah
Professional.  

%\vfill
%
%\selectlanguage{english}
%\pdfbookmark{Abstract}{Abstract}
%\chapter*{Abstract}
%
%\selectlanguage{italian}


\begin{description}
	\item[{\hyperref[cap:introduzione]{Il primo capitolo}}] descrive l’azienda e il progetto di stage assegnato.
	
	\item[{\hyperref[cap:Tecnologie utilizzate]{Il secondo capitolo}}] descrive e tecnologie utlizzate durante
	tutto il periodo di stage.
	
	\item[{\hyperref[cap:analisi-requisiti]{Il terzo capitolo}}] formalizza tutti i casi d’uso e i requisiti ad alto livello raccolti in fase di
	analisi dei requisiti.
	
	\item[{\hyperref[cap:progettazione]{Il quarto capitolo}}] illustra ad alto livello la progettazione.
	
		\item[{\hyperref[cap:progetto-terminato]{Il quinto capitolo}}] illustra ad il progetto realizzato.
	
	\item[{\hyperref[cap:conclusioni]{Il sesto capitolo}}] illustra il prodotto terminato
	
	\item[{\hyperref[cap:conclusioni]{Nel ultimo capitolo}}] viene riportato il glossario
\end{description}
Riguardo la stesura del testo, relativamente al documento sono state adottate le seguenti convenzioni tipografiche:
\begin{itemize}
	\item gli acronimi, le abbreviazioni e i termini ambigui o di uso non comune menzionati vengono definiti nel glossario, situato alla fine del presente documento;
	\item per la prima occorrenza dei termini riportati nel glossario viene utilizzata la seguente nomenclatura: \emph{parola}\glsfirstoccur;
	\item i termini in lingua straniera o facenti parti del gergo tecnico sono evidenziati con il carattere \emph{corsivo}.
\end{itemize}
\endgroup			

\vfill

