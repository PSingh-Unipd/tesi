% !TEX encoding = UTF-8
% !TEX TS-program = pdflatex
% !TEX root = ../tesi.tex

%**************************************************************
% Sommario
%**************************************************************
\cleardoublepage
\phantomsection
\pdfbookmark{Sommario}{Sommario}
\begingroup
\let\clearpage\relax
\let\cleardoublepage\relax
\let\cleardoublepage\relax

\chapter*{Sommario}
Il presente documento descrive il lavoro svolto durante il periodo di stage(dal 03/06/2018 al 03/08/2018), della durata di circa 320 ore, dal laureando Singh Parwinder presso l'azienda Nextep alla sede di Cittadella.
\\
In questo documento verrano descritte in dettaglio l'analisi dei requisiti, la progettazone, l'implementazione e la validazione dell'applicazione CS-template. L'applicazione in questione è stato realizzato utilizzanod le tecnolgie web inovvative sia per quanto riguarda lato frant-end dell'applicazione che quello back-end.\\
L'intero lavoro è stato svolto in ambiente Linux Ubuntu 18.04 LTS. Tutti i diagrammi delle classi, dei package e dei casi d’uso (presenti nei Capitoli 3 e
4) sono conformi allo standard UML 2.0. Per realizzarli è stato usato il software Astah
Professional.  

%\vfill
%
%\selectlanguage{english}
%\pdfbookmark{Abstract}{Abstract}
%\chapter*{Abstract}
%
%\selectlanguage{italian}

\endgroup			

\vfill

