% !TEX encoding = UTF-8
% !TEX TS-program = pdflatex
% !TEX root = ../tesi.tex

%**************************************************************
\chapter{Appendice A}
\section{Servizi di Angular 6}
 Un servizio in Angular 6 è una classe singlethon che implementa funzionalità condivise dai vari elementi di un’applicazione, siano essi componenti che altri servizi.
Ne esiste solo un istanza di uno servizio per tutta l'applicazione, in questo modo i dati condivisi dai diversi componenti sono gli stessi.

L'implementazione di un servizio aviene utilizando il decorator @Injectable() di Angular. Questo decorator aggiunge alla classe dei metadati che permettono ad Angular di iniettare il servizio nei compoenti come una dipendenza. 

Viene utilizzato Dependency injection(DI) design pattern per iniettare le dipendenze dei servizi nei componenti. 
\begin{figure}[!h] 
	\centering 
	\includegraphics[width=0.5\columnwidth]{DI} 
	\caption{DI in Angular}
\end{figure}
Dependency injection è utilizzato ovunque nel Framework Angular. In un'applicazione Angular un componente in media consuma molti servizi utilizzando DI, che li fornisce accesso a tutti i metodi e ai dati forniti dal servizio. 