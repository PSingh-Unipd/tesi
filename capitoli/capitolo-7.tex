% !TEX encoding = UTF-8
% !TEX TS-program = pdflatex
% !TEX root = ../tesi.tex

%**************************************************************
\chapter{Conclusioni}
\label{cap:conclusioni}
%**************************************************************
Questa sezione è dedicata alle valutazioni e considerazioni finali effettuate
al termine del periodo di Stage
%**************************************************************
\section{Raggiungimento degli obiettivi}
Gli obiettivi descritti nella sezione 4.5 dedicata al tracciamento dei requisiti
individuati durante il periodo di Analisi, hanno avuto il seguente risultato:
\begin{center}
	\begin{table}[h]
		\centering
		\begin{tabular}{|c|c|c|ll}
			\cline{1-3}
			\textbf{Tipo}          & \textbf{Individuati} & \textbf{Soddisfatti} &  &  \\ \cline{1-3}
			Requisiti totali       & 16                   & 14                   &  &  \\
			\cline{1-3}
			Requisiti obbligatori & 11                    & 11                    &  &  \\
			\cline{1-3}
			Requisiti desiderabili & 2                    & 0                    &  &  \\ \cline{1-3}
			Requisiti di vincolo   & 3                    & 3                    &  &  \\ \cline{1-3}
		\end{tabular}
	\caption{Soddisfacimento Requisiti}
	\end{table}
	\label{tab:Soddisfacimento Requisiti}

\end{center}
	Durante lo stage ho acquisito diverse conoscenze che mi hanno formato
e portato al raggiungimento degli obiettivi prefissati. Di seguito verranno
elencate le conoscenze principali che hanno caratterizzato il periodo svolto
in azienda.
\begin{itemize}
	\item \textbf{Angular:} ho avuto modo di studiare molto bene il framework Angular. Le SPA sono sempre più richieste nella realtà aziendale. 
	\item \textbf{REST API:} ho avuto modo capire molto bene come funzionano le REST API. 
	\item \textbf{AWS:} sicuramente la tecnologia più interessante sono stati i servizi web di Amazon. Sempre più aziende puntano a queste tecnologie. Si sente sempre più parlare dell'architettura serverless utilizzando in frontend Angular, React ecc.
	\item \textbf{Lavoro aziendale:}  ho imparato cosa significa lavorare in un’azienda,
	la suddivisione dei ruoli e l’importanza del lavoro in team. Nel
	contempo ho sviluppato anche la capacità di essere autonomo in certe
	circostanze, in modo da non dipendere costantemente da altre persone.
\end{itemize}
\section{Valutazione personale sullo stage}
In conclusione l’esperienza che ho vissuto durante il periodo di stage è stata
molto formativa ed interessante, soprattutto perché ho potuto affrontare
nuove tematiche rispetto al mio corso di studi in Università. Nonostante
ciò, grazie alle conoscenze apprese durante gli anni di studio, sono riuscito ad
apprendere velocemente quanto necessario per svolgere il progetto. Inoltre,
ho messo alla prova le mie capacità confrontandomi, per la prima volta, con
il mondo del lavoro e posso affermare di essere rimasto molto soddisfatto. Mi è stato fornito tutto il materiale ed
aiuto necessari per svolgere al meglio il mio lavoro ed ho potuto stabilire un
buon rapporto, lavorativo ed umano, con gli altri dipendenti dell’azienda. Al
termine dello stage ho acquisito consapevolezza delle mie capacità e ritengo
che sia stato un ottimo periodo formativo.
