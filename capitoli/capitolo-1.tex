% !TEX encoding = UTF-8
% !TEX TS-program = pdflatex
% !TEX root = ../tesi.tex

%**************************************************************
\chapter{Introduzione}
\label{cap:introduzione}
In questo capitolo viene brevemente descritta l’azienda ospitante in cui è stata
svolta l’attività di stage e una descrizione ad alto livello del progetto realizzato.

%\noindent Esempio di utilizzo di un termine nel glossario \\
%\gls{api}. \\

%\noindent Esempio di citazione in linea \\
%\cite{site:agile-manifesto}. \\

%\noindent Esempio di citazione nel pie' di pagina \\
%citazione\footcite{womak:lean-thinking} \\

%**************************************************************
\section{Dominio aziendale}
\subsection{L'azienda ospitante}
Nextep è una società fondata nel 2000 da Marco De Toni e Mirco Soffia, con sede
attuale a Cittadella (PD).
Opera nel settore informatico e si occupa di servizi web, web marketing e di infrastrutture
per gestire le informazioni delle aziende, e più in generale ha come obiettivo
quello di migliorare l’efficacia delle strategie di comunicazione web, delle aziende,
dedicando particolare attenzione alla reputazione e all’identità digitale. \\ \\
Nextep fa parte del gruppo Allos, insieme ad Allos Italia, Allos Sud Africa, Allos
USA e Zero12.
Allos si occupa di progetti e tecnologie per lo sviluppo del capitale umano, mentre
Zero12 si occupa dello sviluppo di soluzioni mobile e cloud based.
Il gruppo Allos è stato recentemente acquisito da EOH Holdings Ltd, una grande
società sudafricana.\\ \\
Nextep ha un organico di circa venti persone, tra dipendenti e collaboratori, con
varie competenze: grafici, sviluppatori, esperti di web marketing e tecnici. Sono presenti
tre gruppi principali di lavoro: quello di sviluppo, quello creativo e quello del supporto
tecnico.
In Nextep c’è un ambiente di lavoro giovane, dinamico ma allo stesso tempo professionale,
ed è incentivata la collaborazione e la condivisione di conoscenze e idee
tra le persone. Tutto questo favorisce sia la crescita individuale, dal punto di vista
professionale, che la crescita e l’amalgamazione dei vari gruppi di lavoro.
\begin{figure}[!h] 
	\centering 
	\includegraphics[width=0.7\columnwidth]{azienda} 
	\caption{Logo di Nextep: immagine tratta dal sito dell’azienda}
\end{figure}
%**************************************************************
\subsection{Prodotti e servizi}
Nextep lavora per clienti di diversa tipologia e conformazione, dalla piccola impresa
privata alla multinazionale che si sta espandendo ulteriormente, e con questo offre
svariati prodotti e servizi in base alle esigenze e alle opportunità del mercato e proprie. 
\\ \\La maggior parte dei progetti riguarda la realizzazione di portali e siti web, ma vengono
sviluppati anche diversi altri prodotti, tra cui soluzioni e-commerce e applicazioni
mobile, sviluppo di progetti di virtualizzazione, e storage networking. Inoltre negli ultimi mesi l'azienda si sta dedicato molto anche ai prodotti di machine learning, come i chatbot. \\ \\Nextep offre diversi tipi di servizi tra questi l'installazione e assistenza del portale di customer service Zendesk. Guida le diverse società(piccole o grandi) verso la gestione dei proprio cliente in manienra semplice ed efficace. 
%**************************************************************

\section{Lo stage}
Il progetto di stage è consistito principalmente nella realizzaizone di una applicazione per la piattaforma di customer service Zendesk. La piattaforma Zendesk permette ad un'aziedna di gestire tutte le richieste(in sottoforma di tickets) di propri clienti in unico posto. \\ L'applicazione realizzata permette agli agenti(persone che gestiscono le richieste dei clienti) e agli aministratori di Zendesk di realizzare dei contenuti(chiamati template) HTML e CSS in maniera molto semplice e veloce, ovvero utilizzando un editor drag-and-drop. I template successivamente sono utilizzati nelle risposte verso i clienti. Questo permette di risparmiare una notevole quantità di tempo e non è neccario avere conoscenza di HTML e CSS. Diverse aziende(clienti di Nextep) avevano fatta la rischiesta esplicitamente di tale applicazione. Semplicemnte anche per rispondere agli utenti con le risposte molto "decorate" e velocizzare la generazione di codice html.
\section{Organizzazione del testo}

\begin{description}
    \item[{\hyperref[cap:processi-metodologie]{Il secondo capitolo}}] descrive ...
    
    \item[{\hyperref[cap:descrizione-stage]{Il terzo capitolo}}] approfondisce ...
    
    \item[{\hyperref[cap:analisi-requisiti]{Il quarto capitolo}}] approfondisce ...
    
    \item[{\hyperref[cap:progettazione-codifica]{Il quinto capitolo}}] approfondisce ...
    
    \item[{\hyperref[cap:verifica-validazione]{Il sesto capitolo}}] approfondisce ...
    
    \item[{\hyperref[cap:conclusioni]{Nel settimo capitolo}}] descrive ...
\end{description}

Riguardo la stesura del testo, relativamente al documento sono state adottate le seguenti convenzioni tipografiche:
\begin{itemize}
	\item gli acronimi, le abbreviazioni e i termini ambigui o di uso non comune menzionati vengono definiti nel glossario, situato alla fine del presente documento;
	\item per la prima occorrenza dei termini riportati nel glossario viene utilizzata la seguente nomenclatura: \emph{parola}\glsfirstoccur;
	\item i termini in lingua straniera o facenti parti del gergo tecnico sono evidenziati con il carattere \emph{corsivo}.
\end{itemize}