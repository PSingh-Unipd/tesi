% !TEX encoding = UTF-8
% !TEX TS-program = pdflatex
% !TEX root = ../tesi.tex

%**************************************************************
\chapter{Introduzione}
\label{cap:introduzione}
In questo capitolo viene brevemente descritta l’azienda ospitante in cui è stata
svolta l’attività di stage.
Viene inoltre descritto ad alto livello il progetto CS-Template.

%\noindent Esempio di utilizzo di un termine nel glossario \\
%\gls{api}. \\

%\noindent Esempio di citazione in linea \\
%\cite{site:agile-manifesto}. \\

%\noindent Esempio di citazione nel pie' di pagina \\
%citazione\footcite{womak:lean-thinking} \\

%**************************************************************
\section{L'azienda}

Nextep è una società fondata nel 2000 da Marco De Toni e Mirco Soffia, con sede
attuale a Cittadella (PD).
Opera nel settore informatico e si occupa di servizi web, web marketing e di infrastrutture
per gestire le informazioni delle aziende, e più in generale ha come obiettivo
quello di migliorare l’efficacia delle strategie di comunicazione web, delle aziende,
dedicando particolare attenzione alla reputazione e all’identità digitale.
Nextep fa parte del gruppo Allos, insieme ad Allos Italia, Allos Sud Africa, Allos
USA e Zero12.
Allos si occupa di progetti e tecnologie per lo sviluppo del capitale umano, mentre
Zero12 si occupa dello sviluppo di soluzioni mobile e cloud based.
Il gruppo Allos è stato recentemente acquisito da EOH Holdings Ltd, una grande
società sudafricana.
Nextep ha un organico di circa venti persone, tra dipendenti e collaboratori, con
varie competenze: grafici, sviluppatori, esperti di web marketing e tecnici. Sono presenti
tre gruppi principali di lavoro: quello di sviluppo, quello creativo e quello del supporto
1
2 CAPITOLO 1. L’AZIENDA
tecnico.
In Nextep c’è un ambiente di lavoro giovane, dinamico ma allo stesso tempo professionale,
ed è incentivata la collaborazione e la condivisione di conoscenze e idee
tra le persone. Tutto questo favorisce sia la crescita individuale, dal punto di vista
professionale, che la crescita e l’amalgamazione dei vari gruppi di lavoro.

%**************************************************************
\section{L'idea}

Introduzione all'idea dello stage.

%**************************************************************
\section{Organizzazione del testo}

\begin{description}
    \item[{\hyperref[cap:processi-metodologie]{Il secondo capitolo}}] descrive ...
    
    \item[{\hyperref[cap:descrizione-stage]{Il terzo capitolo}}] approfondisce ...
    
    \item[{\hyperref[cap:analisi-requisiti]{Il quarto capitolo}}] approfondisce ...
    
    \item[{\hyperref[cap:progettazione-codifica]{Il quinto capitolo}}] approfondisce ...
    
    \item[{\hyperref[cap:verifica-validazione]{Il sesto capitolo}}] approfondisce ...
    
    \item[{\hyperref[cap:conclusioni]{Nel settimo capitolo}}] descrive ...
\end{description}

Riguardo la stesura del testo, relativamente al documento sono state adottate le seguenti convenzioni tipografiche:
\begin{itemize}
	\item gli acronimi, le abbreviazioni e i termini ambigui o di uso non comune menzionati vengono definiti nel glossario, situato alla fine del presente documento;
	\item per la prima occorrenza dei termini riportati nel glossario viene utilizzata la seguente nomenclatura: \emph{parola}\glsfirstoccur;
	\item i termini in lingua straniera o facenti parti del gergo tecnico sono evidenziati con il carattere \emph{corsivo}.
\end{itemize}