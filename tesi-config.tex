%**************************************************************
% file contenente le impostazioni della tesi
%**************************************************************

%**************************************************************
% Frontespizio
%**************************************************************

% Autore
\newcommand{\myName}{Riccardo Montagnin}                                    
\newcommand{\myTitle}{Titolo della tesi}

% Tipo di tesi                   
\newcommand{\myDegree}{Tesi di laurea triennale}

% Università             
\newcommand{\myUni}{Università degli Studi di Padova}

% Facoltà       
\newcommand{\myFaculty}{Corso di Laurea in Informatica}

% Dipartimento
\newcommand{\myDepartment}{Dipartimento di Matematica "Tullio Levi-Civita"}

% Titolo del relatore
\newcommand{\profTitle}{Prof.}

% Relatore
\newcommand{\myProf}{Tullio Vardanega}

% Luogo
\newcommand{\myLocation}{Padova}

% Anno accademico
\newcommand{\myAA}{2016-2017}

% Data discussione
\newcommand{\myTime}{Luglio 2017}


%**************************************************************
% Impostazioni di impaginazione
% see: http://wwwcdf.pd.infn.it/AppuntiLinux/a2547.htm
%**************************************************************

\setlength{\parindent}{14pt}   % larghezza rientro della prima riga
\setlength{\parskip}{0pt}   % distanza tra i paragrafi


%**************************************************************
% Impostazioni di biblatex
%**************************************************************
\bibliography{bibliografia} % database di biblatex 

\defbibheading{bibliography} {
    \cleardoublepage
    \phantomsection 
    \addcontentsline{toc}{chapter}{\bibname}
    \chapter*{\bibname\markboth{\bibname}{\bibname}}
}

\setlength\bibitemsep{1.5\itemsep} % spazio tra entry

\DeclareBibliographyCategory{opere}
\DeclareBibliographyCategory{web}

\addtocategory{opere}{womak:lean-thinking}
\addtocategory{web}{site:agile-manifesto}

\defbibheading{opere}{\section*{Riferimenti bibliografici}}
\defbibheading{web}{\section*{Siti Web consultati}}


%**************************************************************
% Impostazioni di caption
%**************************************************************
\captionsetup{
    tableposition=top,
    figureposition=bottom,
    font=small,
    format=hang,
    labelfont=bf
}

%**************************************************************
% Impostazioni di glossaries
%**************************************************************

\usepackage{glossaries}
\makeglossaries
%INIZIO NOME
%\newglossaryentry{nome}{
%	name={nome},
%	description={descrizione}
%}
%FINE NOME

%INIZIO UML
\newglossaryentry{UML}{
	name={UML},
	description={\\ Unified Modeling Language, linguaggio di modellizzazione e specifica basato sul paradigma orientato agli oggetti}
}
\newglossaryentry{API}{
	name={API},
	description={\\ Acronimo di Application Programming Interface (ovvero interfaccia di programmazione
		di una applicazione) si indicano un unsieme di procedure (raggruppate assieme per
		formare un insieme di strumenti specifici) per il compimento di un determinato compito
		all’interno di un certo programma}
}
\newglossaryentry{iframe}{
	name={iframe},
	description={L'iframe in informatica è un elemento HTML. Si tratta infatti di un frame "ancorato" all'interno della pagina, equivale cioè ad un normale frame, ma con la differenza di essere un elemento inline (interno) della pagina, non esterno.
	L'iframe viene generalmente utilizzato per mostrare il contenuto di una pagina web, o di una qualsivoglia risorsa, all'interno di un riquadro in una seconda pagina principale}
}
\newglossaryentry{SPA}{
	name={SPA},
	description={Single Page Application è una metologià per la creazione delle interfacce web. In questo caso esiste una sola pagina HTML, tutto il resto(navigazione, pagine) viene gestito  tramite Javascript. I più famosi framework/librerie sono Angular, React e Vue }
}
\newglossaryentry{framework}{
	name={framework},
	description={Un framework è un'architettura logica di supporto (spesso un'implementazione logica di un particolare design pattern) su cui un software può essere progettato e realizzato, spesso facilitandone lo sviluppo da parte del programmatore. Inoltre fornisce una raccolta di librerie ed oggetti già fatti}
}
\newglossaryentry{MVC}{
	name={MVC},
	description={Model-view-controller in informatica, è un pattern architetturale molto diffuso nello sviluppo di sistemi software, in particolare nell'ambito della programmazione orientata agli oggetti, in grado di separare la logica di presentazione dei dati dalla logica di business}
}
\newglossaryentry{MVVM}{
	name={MVVM},
	description={
	Una variante del pattern MVC è MVVM, Model View ViewModel. Questo pattern propone un ruolo più attivo della View rispetto a MVC: la View è in grado di gestire eventi, eseguire operazioni ed effettuare il data-binding. In questo contesto, quindi, alcune delle funzionalità del Controller vengono inglobate nella View, la quale si appoggia su un’estensione del Model: il ViewModel.
	Il ViewModel è quindi un Model esteso con funzionalità per la manipolazione dei dati e per l’interazione con la View}
}
\newglossaryentry{design pattern}{
	name={design pattern},
	description={Questo termine fa riferimento a una soluzione progettuale generale ad un problema
		ricorrente che può presentarsi in diverse situazioni durante le fasi di progettazione e
		sviluppo del software}
}
\newglossaryentry{W3C Recommendation}{
	name={W3C Recommendation},
	description={La parola W3C significa World Wide Web Consortium, cioè il Consorzio Internazionale che si occupa di definire linee guida e standard non proprietari, le tecnologie, i protocolli e tutto quanto necessario per lo sviluppo del Web }
}
\newglossaryentry{Javascript}{
	name={Javascript},
	description={JavaScript è un linguaggio di scripting orientato agli oggetti e agli eventi, comunemente utilizzato nella programmazione Web lato client per la creazione, in siti web e applicazioni web, di effetti dinamici interattivi tramite funzioni di script invocate da eventi innescati a loro volta in vari modi dall'utente sulla pagina web in uso}
}
\newglossaryentry{EMCAScript 6}{
	name={EMCAScript 6},
	description={Spesso conoscitto con acronimo ES6 è una versione del linguaggio di scripting Javascript standardizzata }
}
\newglossaryentry{UI}{
	name={UI},
	description={User Interface è un'interfaccia uomo-macchina, ovvero ciò che si frappone tra una macchina e un utente, consentendone l'interazione reciproca}
}
\newglossaryentry{serverless}{
	name={serverless},
	description={Nel contesto dell'applicazone, serverless è un metodo di creazione ed esecuzione di applicazioni e servizi che non richiede la gestione dell'infrastruttura. L'applicazione sarà comunque eseguita su server, ma la gestione di quest'ultimo sarà a carico di AWS }
}



\newglossaryentry{mock}{
	name={mock},
	description={Dal inglese "finto". Utilizzato nello sviluppo per simulare il comportamento degli oggetti reali, in questo modo è possibile provare delle parti di un sistema senza necessariamente implementare tutto}
}

\newglossaryentry{repository}{
	name={repository},
	description={E' un ambiente condiviso di sviluppo software. In generale è una cartella condivisa da molti utenti nel quale è posibile lavorare insieme in maniera molto semplice}
}
\newglossaryentry{SDK}{
	name={SDK},
	description={Acronimo di Software Development Kit, si riferisce genericamente a un insieme di
		strumenti per migliorare lo sviluppo o la documentazione di software}
}



\newglossaryentry{JSON}{
	name={JSON},
	description={Acronimo di JavaScript Object Notation, è un formato dichiarativo basato sullo
	Standard ECMA-262 adatto all’interscambio di dati fra applicazioni client-server}
}


\newglossaryentry{Observer}{
	name={Observer},
	description={E' un design patter che sta alla base di MVC, permette a molti oservatori di oservare un unica fonte di verità. In questo modo tutti gli oggetti hanno i stessi dati}
}


\newglossaryentry{Decorator}{
	name={Decorator},
	description={Decorator è un design pattern che permette di decorare un'oggetto utilizzando le funzioni d'altri oggetti. Utilizzato per diminuire l'esagerato utilizzato dell' ereditarietà. }
}

\newglossaryentry{Singleton}{
	name={Singleton},
	description={Singleton è un design pattern che permette di creare un unica isatanza di un'oggetto in tutta l'applicazione}
}

\newglossaryentry{DI}{
	name={DI},
	description={Acronimo di Dependency Injection, è un design pattern che risolve il problema di dipendenze. Utilizzato soprattutto nella programmazione ad oggetti, dove spesso molti oggetti contengono altri oggetti come parametri. Questo pattern permette di iniettare le dipendeze tra gli oggetti invece di avere le istanze interne, in questo modo ogni oggetto può essere modificato senza causare danni in altri oggetti}
} % database di termini
\makeglossaries


%**************************************************************
% Impostazioni di graphicx
%**************************************************************
\graphicspath{{immagini/}} % cartella dove sono riposte le immagini


%**************************************************************
% Impostazioni di hyperref
%**************************************************************
\hypersetup{
    %hyperfootnotes=false,
    %pdfpagelabels,
    %draft,	% = elimina tutti i link (utile per stampe in bianco e nero)
    colorlinks=true,
    linktocpage=true,
    pdfstartpage=1,
    pdfstartview=FitV,
    % decommenta la riga seguente per avere link in nero (per esempio per la stampa in bianco e nero)
    %colorlinks=false, linktocpage=false, pdfborder={0 0 0}, pdfstartpage=1, pdfstartview=FitV,
    breaklinks=true,
    pdfpagemode=UseNone,
    pageanchor=true,
    pdfpagemode=UseOutlines,
    plainpages=false,
    bookmarksnumbered,
    bookmarksopen=true,
    bookmarksopenlevel=1,
    hypertexnames=true,
    pdfhighlight=/O,
    %nesting=true,
    %frenchlinks,
    urlcolor=webbrown,
    linkcolor=RoyalBlue,
    citecolor=webgreen,
    %pagecolor=RoyalBlue,
    %urlcolor=Black, linkcolor=Black, citecolor=Black, %pagecolor=Black,
    pdftitle={\myTitle},
    pdfauthor={\textcopyright\ \myName, \myUni, \myFaculty},
    pdfsubject={},
    pdfkeywords={},
    pdfcreator={pdfLaTeX},
    pdfproducer={LaTeX}
}

%**************************************************************
% Impostazioni di itemize
%**************************************************************
\renewcommand{\labelitemi}{$\ast$}

%\renewcommand{\labelitemi}{$\bullet$}
%\renewcommand{\labelitemii}{$\cdot$}
%\renewcommand{\labelitemiii}{$\diamond$}
%\renewcommand{\labelitemiv}{$\ast$}


%**************************************************************
% Impostazioni di listings
%**************************************************************
\lstset{
    language=[LaTeX]Tex,%C++,
    keywordstyle=\color{RoyalBlue}, %\bfseries,
    basicstyle=\small\ttfamily,
    %identifierstyle=\color{NavyBlue},
    commentstyle=\color{Green}\ttfamily,
    stringstyle=\rmfamily,
    numbers=none, %left,%
    numberstyle=\scriptsize, %\tiny
    stepnumber=5,
    numbersep=8pt,
    showstringspaces=false,
    breaklines=true,
    frameround=ftff,
    frame=single
} 


%**************************************************************
% Impostazioni di xcolor
%**************************************************************
\definecolor{webgreen}{rgb}{0,.5,0}
\definecolor{webbrown}{rgb}{.6,0,0}


%**************************************************************
% Altro
%**************************************************************

\newcommand{\omissis}{[\dots\negthinspace]} % produce [...]

% eccezioni all'algoritmo di sillabazione
\hyphenation
{
    ma-cro-istru-zio-ne
    gi-ral-din
}

\newcommand{\sectionname}{sezione}
\addto\captionsitalian{\renewcommand{\figurename}{Figura}
                       \renewcommand{\tablename}{Tabella}}

\newcommand{\glsfirstoccur}{\ap{{[g]}}}

\newcommand{\intro}[1]{\emph{\textsf{#1}}}

%**************************************************************
% Environment per ``rischi''
%**************************************************************
\newcounter{riskcounter}                % define a counter
\setcounter{riskcounter}{0}             % set the counter to some initial value

%%%% Parameters
% #1: Title
\newenvironment{risk}[1]{
    \refstepcounter{riskcounter}        % increment counter
    \par \noindent                      % start new paragraph
    \textbf{\arabic{riskcounter}. #1}   % display the title before the 
                                        % content of the environment is displayed 
}{
    \par\medskip
}

\newcommand{\riskname}{Rischio}

\newcommand{\riskdescription}[1]{\textbf{\\Descrizione:} #1.}

\newcommand{\risksolution}[1]{\textbf{\\Soluzione:} #1.}

%**************************************************************
% Environment per ``use case''
%**************************************************************
\newcounter{usecasecounter}             % define a counter
\setcounter{usecasecounter}{0}          % set the counter to some initial value

%%%% Parameters
% #1: ID
% #2: Nome
\newenvironment{usecase}[2]{
    \renewcommand{\theusecasecounter}{\usecasename #1}  % this is where the display of 
                                                        % the counter is overwritten/modified
    \refstepcounter{usecasecounter}             % increment counter
    \vspace{10pt}
    \par \noindent                              % start new paragraph
    {\large \textbf{\usecasename #1: #2}}       % display the title before the 
                                                % content of the environment is displayed 
    \medskip
}{
    \medskip
}

\newcommand{\usecasename}{UC}

\newcommand{\usecaseactors}[1]{\textbf{\\Attori Principali:} #1. \vspace{4pt}}
\newcommand{\usecasepre}[1]{\textbf{\\Precondizioni:} #1. \vspace{4pt}}
\newcommand{\usecasedesc}[1]{\textbf{\\Descrizione:} #1. \vspace{4pt}}
\newcommand{\usecasepost}[1]{\textbf{\\Postcondizioni:} #1. \vspace{4pt}}
\newcommand{\usecasealt}[1]{\textbf{\\Scenario Alternativo:} #1. \vspace{4pt}}

%**************************************************************
% Environment per ``namespace description''
%**************************************************************

\newenvironment{namespacedesc}{
    \vspace{10pt}
    \par \noindent                              % start new paragraph
    \begin{description} 
}{
    \end{description}
    \medskip
}

\newcommand{\classdesc}[2]{\item[\textbf{#1:}] #2}